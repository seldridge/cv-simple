\documentclass{article}

\usepackage{hyperref}
\hypersetup {
  pdfauthor={Schuyler Eldridge},
  pdftitle={Schuyler Eldridge -- CV},
}

\usepackage[
  top=1in,
  bottom=1in,
  left=1in,
  right=1in,
]{geometry}

\usepackage{fancyhdr}
\pagestyle{fancy}
\newcommand\header{
  \lhead{\textsc{Schuyler Eldridge}}
  \chead{\textsc{Curriculum Vitae}}
  \rhead{\textsc{Last Updated \today}}
  \renewcommand{\headrulewidth}{0.5pt} }
\newcommand\firstFooter{
  \lfoot{\href{https://github.com/seldridge}{\textsc{seldridge@github}}}
  \cfoot{}
  \rfoot{\href{mailto:schuyler.eldridge@gmail.com}{\textsc{schuyler.eldridge@gmail.com}}}
  \renewcommand{\footrulewidth}{0.5pt} }
\newcommand\otherFooter{
  \lfoot{}
  \cfoot{\thepage}
  \rfoot{}
  \renewcommand{\footrulewidth}{0pt} }

\usepackage{datetime}
\newdateformat{yyyymmdate}{\twodigit{\THEMONTH}/\twodigit{\THEYEAR}}

\usepackage{multirow}

\usepackage{xspace}

\usepackage{booktabs}

\usepackage[
  backend=biber,
  maxbibnames=1024,
  sorting=ydnt,
  defernumbers=true,
]{biblatex}
\addbibresource{schuyler.bib}

\usepackage{xcolor}
\definecolor{Greens9p6}{rgb}{0.137254901960784,0.545098039215686,0.270588235294118}
\definecolor{Reds9p7}{rgb}{0.647058823529412,0.0588235294117647,0.0823529411764706}

\usepackage{tabu}

\usepackage{longtable}
\setlength\extrarowheight{1ex}

\newcommand{\riscv}{RISC-V\xspace}
\newcommand{\ibmTjWatson}{IBM T.\ J.\ Watson Research Center\xspace}

\begin{document}
\yyyymmdddate
\header\firstFooter

\subsection*{Education}
\newcommand\entry[5]{#1, #2 & #3 & #4, #5\\}
\begin{tabu} to \textwidth {lX[c]r}
  \entry{Boston University}{Boston, MA}{Computer Engineering}{Ph.~D.}{2016}
  \entry{Boston University}{Boston, MA}{Electrical Engineering}{B.~S.}{2010}
\end{tabu}

\subsection*{Employment History}
\renewcommand\entry[4]{#1 & \multirow{2}{*}{\shortstack[c]{#2\\#3}} & {\yyyymmdate #4}\\\\}
\begin{longtabu}{lX[c]r}
  \entry{Research Staff Member}{Reliability and Power Aware Microarchitectures}{\ibmTjWatson, NY}{\formatdate{02}{02}{2019}--present}
  \entry{Postdoctoral Researcher}{Reliability and Power Aware Microarchitectures}{\ibmTjWatson, NY}{\formatdate{27}{8}{16}--\formatdate{02}{02}{2019}}
  \entry{Space Technology Research Fellow}{Robotic Systems Estimation, Decision, and Control}{NASA Jet Propulsion Lab, Pasadena, CA}{\multirow{3}{*}{\shortstack{\formatdate{04}{05}{15}--\formatdate{01}{07}{15} \\ \formatdate{02}{06}{14}--\formatdate{22}{08}{14} \\ \formatdate{03}{06}{13}--\formatdate{23}{08}{13}}}}\\
  \entry{Graduate Technical Intern}{Server Memory Controller Validation}{Intel Corporation, Hudson, MA}{\formatdate{16}{05}{11}--\formatdate{18}{09}{11}}
  \entry{Graduate Technical Intern}{Data Center and Connected Systems}{Intel Corporation, Hudson, MA}{\formatdate{20}{07}{10}--\formatdate{27}{08}{10}}
  \entry{Undergraduate Teaching Assistant}{Undergraduate Logic Design}{Boston University}{\multirow{2}{*}{\shortstack{\formatdate{01}{01}{10}--\formatdate{01}{05}{10} \\ \formatdate{01}{01}{09}--\formatdate{01}{05}{09}}}}
\end{longtabu}

\subsection*{Honors, Awards, and Fellowships}
\renewcommand\entry[2]{#1 & {\yyyymmdate #2}\\}
\begin{longtabu} to \textwidth {X[l]r}
  \entry{NASA Space Technology Research Fellowship}{\formatdate{01}{08}{12}--\formatdate{01}{08}{16}}
  \entry{CELEST/CompNet Award}{\yyyymmdddate \formatdate{23}{04}{2012}}
  \entry{Boston University Dean's Fellowship}{\formatdate{08}{09}{10}--\formatdate{08}{09}{11}}
  \entry{P.~T.~Hsu Memorial Award for Outstanding Senior Design Project}{\yyyymmdddate\formatdate{03}{05}{2010}}
  \entry{Boston University Engineering Scholar Award}{\formatdate{01}{09}{06}--\formatdate{15}{05}{10}}
\end{longtabu}

\subsection*{Grants}
\renewcommand\entry[4]{#1 & #2 & #3 & {#4}\\}
\begin{longtabu} to \textwidth {X[l]lllr}
  \entry{PERFECT: Efficient Resilience in Embedded Computing}{DARPA}{Staff}{\formatdate{27}{08}{2016}--\formatdate{31}{08}{2018}}
  \entry{DSSoC: Domain Specific System on Chip}{DARPA}{Staff}{\formatdate{01}{08}{2018}--present}
\end{longtabu}

\subsection*{Program Committees and Reviews}
\renewcommand\entry[3]{#1 & #2 & {#3}\\}
\begin{longtabu} to \textwidth {Xlr}
  \entry{Chisel Community Conference (CCC)}{Program Committee}{2018}
  \entry{IEEE Micro}{Article Reviewer}{2018}
  \entry{IEEE Micro}{Article Reviewer}{2016}
\end{longtabu}

\newpage
\otherFooter

\subsection*{Open Source Activities (GitHub)}

\subsubsection*{Maintainer}
\renewcommand\entry[5]{\href{https://github.com/#1/#2}{\texttt{#1/#2}} & #3 & #4 & \ifthenelse{\equal{#5}{0}}{}{#5 Stars}\\}
\begin{longtabu} to \textwidth {llXr}
  \entry{seldridge}{verilog}{Verilog}{Repository for basic (and not so basic) Verilog blocks with high re-use potential}{227}
  \entry{bu-icsg}{dana}{Scala}{Dynamically Allocated Neural Network Accelerator for the \riscv Rocket Microprocessor in Chisel}{143}
  \entry{seldridge}{rocket-rocc-examples}{C}{Tests for example Rocket Custom Coprocessors}{38}
  \entry{IBM}{hdl-tools}{Tcl}{Facilitates building open source tools for working with hardware description languages (HDLs)}{29}
  \entry{IBM}{rocc-software}{C}{C/Assembly macros for talking with Rocket Custom Coprocessors (RoCCs)}{27}
  \entry{IBM}{chiffre}{Scala}{A fault-injection framework using Chisel and FIRRTL}{21}
  \entry{IBM}{perfect-chisel}{Scala}{Chisel artifacts developed under IBM's involvement with the DARPA PERFECT program}{16}
  \entry{IBM}{firrtl-mode}{Emacs Lisp}{Major mode for editing FIRRTL files in Emacs}{1}
  \entry{seldridge}{make-markdown}{Shell}{Collecting personal knowledge in Markdown}{0}
\end{longtabu}

\subsubsection*{Contributor}
\renewcommand\entry[7]{\href{https://github.com/#1/#2}{\texttt{#1/#2}} & #3 & #4 & #5 \ifthenelse{\equal{#5}{1}}{commit}{commits} \textcolor{Greens9p6}{\texttt{#6++}} \textcolor{Reds9p7}{\texttt{#7--}}\\}
\begin{longtabu} to \textwidth {llX[l]r}
  \entry{freechipsproject}{chisel3}{Scala}{Constructing Hardware in a Scala Embedded Language version 3}{115}{5827}{1996}
  \entry{freechipsproject}{firrtl}{Scala}{Flexible Intermediate Representation for RTL}{106}{10600}{3144}
  \entry{freechipsproject}{rocket-chip}{Scala}{Rocket Chip Generator}{30}{721}{234}
  \entry{riscv}{riscv-fesvr}{C}{\riscv Frontend Server}{9}{298}{148}
  \entry{riscv}{riscv-tools}{Shell}{\riscv Tools (GNU Toolchain, ISA Simulator, Tests)}{3}{3}{3}
  \entry{ucb-bar}{chisel2-deprecated}{Scala}{Constructing Hardware in a Scala Embedded Language version 2}{2}{17}{3}
  \entry{freechipsproject}{chisel-testers}{Scala}{Provides various testers for chisel users}{2}{12}{4}
  \entry{ucb-bar}{generator-bootcamp}{Scala}{Generator Bootcamp Material: Learn Chisel the Right Way}{2}{8}{8}
  \entry{ucb-bar}{chisel-testers2}{Scala}{ChiselTest}{1}{1}{1}
  \entry{ccelio}{riscv-boom-doc}{\LaTeX}{Documentation for the BOOM processor}{1}{1}{1}
  \entry{melpa}{melpa}{Emacs Lisp}{Recipes and build machinery for the biggest Emacs package repo}{1}{3}{0}
\end{longtabu}

\subsection*{Publications}
\nocite{*}

\subsubsection*{Peer Reviewed Conference Publications}
\printbibliography[heading=none, resetnumbers=true, keyword={conference}]

\subsubsection*{Peer Reviewed Journal Articles}
\printbibliography[heading=none, resetnumbers=true, keyword={journal}]

\subsubsection*{Patents and Patent Applications}
\printbibliography[heading=none, resetnumbers=true, keyword={patent}]

\subsubsection*{Demonstrations}
\printbibliography[heading=none, resetnumbers=true, keyword={demo}]

\subsubsection*{Technical Reports}
\printbibliography[heading=none, resetnumbers=true, type={report}]

\subsubsection*{Theses}
\printbibliography[heading=none, resetnumbers=true, keyword={thesis}]

\subsection*{Workshop Talks and Posters}
\printbibliography[heading=none, resetnumbers=true, keyword={workshop}]

\subsection*{Panel Participation}
\renewcommand\entry[5]{\multirow{2}{*}{\shortstack[l]{#1: ``#2''\\\quad#3}} & \multirow{2}{*}{\shortstack[r]{#4\\#5}}\\\\}
\begin{longtabu} to \textwidth {Xr}
  \entry{Panelist}{Open Discussion -- Current State of \riscv Research}{1st Workshop on Computer Architecture Research with \riscv (CARRV)}{\formatdate{14}{10}{2017}}{Boston, MA}
  \entry{Panelist}{Building Efficient and Resilient AI Systems}{3rd Workshop on Cognitive Architectures}{\formatdate{24}{03}{2018}}{Williamsburg, VA}
\end{longtabu}

\subsection*{Thesis Committees}
\renewcommand\entry[5]{#1 & #2 & #3 & #4 & #5}
\begin{longtabu} to \textwidth {lllXr}
  \entry{$4^\textrm{th}$ Reader}{Ph.~D.}{University of Virginia}{Alec Roelke, ``Improving Reliability and Security with Aging and Pre-RTL Modeling''}{2018}
\end{longtabu}

\subsection*{Doctoral Advisor}
Ajay Joshi (Boston University)

\end{document}
